\section*{Memo to Producers of Dancing with the Stars}

\subsection*{Executive Summary}

After comprehensive comprehensive statistical analysis of 34 seasons of DWTS data, encompassing 299 elimination weeks and hundreds of contestants, we recommend adopting a \textbf{percent-based voting method} for future seasons, and strongly suggest implementing our proposed \textbf{machine learning-based intelligent voting system} for optimal results. Our comprehensive statistical analysis demonstrates that the percent-based method achieves 97.0\% accuracy compared to 60.5\% for the rank-based method, while the proposed ML system achieves 97.99\% accuracy, representing a 9.03 percentage point improvement over current methods.

\subsection*{Key Findings}

\subsubsection*{Current System Performance}

The current voting system, which alternates between rank-based and percent-based methods, achieves an overall accuracy of 88.96\% when applied retrospectively to all historical data. However, this aggregate figure masks substantial differences between methods. The percent-based method, used in Seasons 3--27, achieves 97.0\% accuracy (290 correct predictions out of 299 weeks), while the rank-based method, used in Seasons 1--2 and 28--34, achieves only 60.5\% accuracy (181 correct predictions). This difference of 36.5 percentage points is statistically statistically significant and indicates that the choice of voting method has a profound impact on prediction accuracy.

The two methods disagree in 37.46\% of elimination weeks (112 out of 299), demonstrating that the choice of method statistically significantly affects outcomes. This high disagreement rate suggests that method selection is not merely a technical detail but a fundamental decision affecting competition results.

\subsubsection*{New System Performance}

The proposed machine learning-based voting system achieves 97.99\% prediction accuracy (293 correct predictions out of 299 weeks), representing a 9.03 percentage point improvement over the current system baseline of 88.96\%. This improvement translates to 27 additional correct predictions, demonstrating the system's superior performance.

Beyond accuracy, the system provides additional benefits: automatic consideration of multiple factors (age, professional dancer, industry, region) promotes fairness, high accuracy reduces controversy by producing more predictable and acceptable eliminations, and maintained emphasis on fan votes (62.7\% importance) preserves competitive intensity and audience engagement.

\subsection*{Why Percent-Based Method?}

The percent-based method demonstrates substantial superiority over the rank-based method for several reasons. First, it achieves statistically significantly higher accuracy (97.0\% versus 60.5\%), providing more reliable predictions of actual elimination outcomes. Second, it better represents fan preferences by allowing high fan vote percentages to compensate for low judge score percentages, enabling popular contestants to survive despite technical deficiencies. Third, it has a proven track record, having been used successfully in Seasons 3--27, during which the show maintained high viewership and audience engagement. Fourth, it handles controversial cases better, as demonstrated by its correct predictions for Jerry Rice, Bristol Palin, and Bobby Bones, all of whom had high fan support but low judge scores.

The mathematical foundation for this superiority lies in the method's sensitivity to vote magnitude. Unlike the rank-based method, which treats all rank positions equally regardless of underlying score or vote differences, the percent-based method directly incorporates the magnitude of differences, making it more responsive to actual voting patterns.

\subsection*{Why ML-Based System?}

The proposed machine learning-based system offers several advantages over both traditional methods. First, it achieves the highest accuracy of all methods tested (97.99\%), providing optimal prediction performance. Second, it automatically promotes fairness by encoding and considering age, professional dancer, industry, and region as features, learning optimal weights that balance these factors without requiring manual specification. Third, it adapts to different season characteristics automatically, learning season-specific patterns that may not be captured by fixed methods. Fourth, it reduces controversy through high accuracy, as more accurate predictions lead to more acceptable eliminations and reduced audience dissatisfaction. Fifth, it maintains excitement by continuing to prioritize fan votes (62.7\% total importance), ensuring that audience participation remains meaningful and competitive intensity is preserved.

The system's interpretability, through feature importance comprehensive statistical analysis, addresses concerns about black-box predictions while maintaining the benefits of machine learning approaches. The ability to explain decisions through feature contributions provides transparency without sacrificing performance.

\subsection*{Implementation Recommendations}

\subsubsection*{Short-term (Next Season)}

For immediate implementation, we recommend switching to the percent-based method if not already in use. This change requires minimal technical modification and can be implemented immediately, providing substantial accuracy improvements with low risk. The "judges choose from bottom two" mechanism introduced in Season 28 should be retained, as it provides additional fairness by allowing expert judgment to override purely mathematical results when appropriate.

A hybrid approach may be considered: use the percent-based method to identify the bottom two contestants, then allow judges to select which of these two will be eliminated. This approach combines the accuracy of the percent-based method with the flexibility of expert judgment, potentially optimizing both fairness and excitement.

\subsubsection*{Long-term (1--2 Seasons)}

For long-term improvement, we recommend piloting the ML-based system in 1--2 seasons with close monitoring and performance evaluation. This phased approach allows for gradual adoption, risk mitigation, and refinement based on real-world feedback. The pilot should include comprehensive monitoring of prediction accuracy, controversy levels, and audience satisfaction, with regular model updates based on new data.

Gradual expansion to full deployment should follow successful pilot results, with continuous monitoring and model updates to ensure optimal performance. The system's adaptive nature allows it to improve over time as more data becomes available, potentially achieving even higher accuracy in future seasons.

\subsection*{Expected Benefits}

Implementation of the recommended changes is expected to yield several benefits. Accuracy improvements of 9.03 percentage points translate to more reliable predictions and reduced uncertainty in elimination outcomes. Automatic consideration of multiple factors reduces bias attributable to age, professional dancer, industry, and region, promoting fairness. More accurate predictions lead to fewer disputed eliminations, reducing controversy and maintaining audience trust. Better balance between fan votes and judge scores ensures that both expert judgment and audience preferences are appropriately weighted, enhancing overall satisfaction.

\subsection*{Risk Assessment}

Several risks must be considered and mitigated. Overfitting risk is low, as the model is validated using time-series cross-validation and regularization, with a training-validation gap of only 0.09 percentage points. Data dependency is a medium risk, as the system requires historical data, but 34 seasons provide a substantial foundation, and transfer learning and online learning capabilities can address data limitations. Explainability is a medium concern, as machine learning models are less transparent than simple methods, but feature importance comprehensive statistical analysis and potential SHAP value computation provide interpretability. System failure risk is low, with backup systems and monitoring in place to ensure reliability.

Mitigation strategies include maintaining the original system as a backup, providing feature importance explanations for transparency, implementing comprehensive monitoring and alerting systems, and establishing clear protocols for system updates and maintenance.

\subsection*{Next Steps}

To proceed with implementation, we recommend the following steps. First, review this comprehensive statistical analysis with the production team to ensure understanding of findings and recommendations. Second, decide on immediate action regarding the percent-based method, considering the substantial accuracy improvements and low implementation risk. Third, plan the ML system pilot program, including resource allocation, timeline, and success metrics. Fourth, allocate resources for implementation, including technical infrastructure, personnel, and budget. Fifth, set up monitoring and evaluation frameworks to track performance, identify issues, and guide continuous improvement.

\subsection*{Conclusion}

The data clearly demonstrates that the percent-based method is superior to the rank-based method in both accuracy and alignment with actual voting behavior. Our proposed ML system offers the best combination of accuracy, fairness, and excitement, achieving 97.99\% accuracy while automatically considering multiple factors and maintaining audience engagement. We strongly recommend adopting these improvements to enhance the show's integrity, reduce controversy, and improve audience satisfaction.

\vspace{1cm}

\textbf{Contact}: For questions or further discussion, please refer to the complete technical report, which provides detailed mathematical formulations, validation procedures, and implementation guidelines.
