\subsection{Problem Background}

Dancing with the Stars (DWTS) represents a prominent television competition format wherein celebrities partner with professional dancers to perform choreographed routines. Expert judges evaluate performances using a standardized scoring system (1--10 scale), while viewers cast votes for their preferred couples through various platforms. The integration of these two components determines weekly elimination outcomes, creating a complex decision-making framework that balances expert judgment with public sentiment.

The competition has completed 34 seasons, during which the voting mechanism has evolved. Seasons 1--2 employed a rank-based aggregation method, combining judge ranks and fan ranks. Seasons 3--27 transitioned to a percent-based method, combining judge score percentages and fan vote percentages. Beginning with Season 28, the competition returned to the rank-based approach while introducing a "judges choose from bottom two" mechanism, allowing expert judgment to override purely mathematical results in specific circumstances.

\subsection{Research Objectives}

This investigation addresses five primary research questions: (1) How can confidential fan votes be accurately estimated from observed elimination outcomes? (2) What are the comparative advantages and disadvantages of rank-based versus percent-based voting aggregation methods? (3) Which voting method better handles controversial cases where popular contestants survive despite low technical scores? (4) What are the differential impacts of various factors (professional dancers, contestant age, industry background, geographic region) on judge scores versus fan votes? (5) Can a machine learning-based voting system achieve superior accuracy while maintaining fairness and competitive excitement?

\subsection{Data Description}

The dataset encompasses comprehensive information for all 34 seasons, comprising 299 elimination weeks and hundreds of contestant-seasons. Key variables include contestant demographics (age, industry, region), professional dancer partnerships, weekly judge scores from multiple evaluators, and final competition placements. Notably, fan votes remain confidential throughout the competition, necessitating estimation through inverse problem methodologies.
