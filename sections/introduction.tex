\subsection{Problem Background}

Dancing with the Stars (DWTS) represents a prominent television competition format wherein celebrities partner with professional dancers to perform choreographed routines. Expert judges evaluate performances using a standardized scoring system (1--10 scale), while viewers cast votes for their preferred couples through various platforms. The integration of these two components determines weekly elimination outcomes, creating a complex decision-making framework that balances expert judgment with public sentiment.

The competition has completed 34 seasons, during which the voting mechanism has evolved. Seasons 1--2 employed a rank-based aggregation method, combining judge ranks and fan ranks. Seasons 3--27 transitioned to a percent-based method, combining judge score percentages and fan vote percentages. Beginning with Season 28, the competition returned to the rank-based approach while introducing a "judges choose from bottom two" mechanism, allowing expert judgment to override purely mathematical results in specific circumstances.

\subsection{Research Objectives}

This investigation addresses the following research questions as specified in the problem statement:

\begin{enumerate}
\item \textbf{Fan Vote Estimation}: Develop a mathematical model to produce estimated fan votes for each contestant for the weeks they competed. Does the model correctly estimate fan votes that lead to results consistent with who was eliminated each week? Provide measures of consistency. How much certainty is there in the fan vote totals produced, and is that certainty always the same for each contestant/week? Provide measures of certainty.

\item \textbf{Voting Method Comparison}: Compare and contrast the results produced by the two approaches (rank-based and percent-based) used by the show to combine judge and fan votes across all seasons. If differences in outcomes exist, does one method seem to favor fan votes more than the other? Examine the two voting methods applied to specific celebrities where there was "controversy". Would the choice of method have led to the same result for each of these contestants? How would including the additional approach of having judges choose which of the bottom two couples to eliminate each week impact the results? Based on analysis, which method would be recommended for future seasons and why? Would you suggest including the additional approach of judges choosing from the bottom two couples?

\item \textbf{Factor Impact Analysis}: Use the data including fan vote estimates to develop a model that analyzes the impact of various pro dancers as well as characteristics for the celebrities (age, industry, etc.). How much do such things impact how well a celebrity will do in the competition? Do they impact judges scores and fan votes in the same way?

\item \textbf{New Voting System Proposal}: Propose another system using fan votes and judge scores each week that is more "fair" (or "better" in some other way such as making the show more exciting for the fans). Provide support for why this approach should be adopted by the show producers.
\end{enumerate}

\subsection{Data Description}

The dataset encompasses comprehensive information for all 34 seasons, comprising 299 elimination weeks and hundreds of contestant-seasons. Key variables include contestant demographics (age, industry, region), professional dancer partnerships, weekly judge scores from multiple evaluators, and final competition placements. Notably, fan votes remain confidential throughout the competition, necessitating estimation through inverse problem methodologies.
