\subsection{Methodology}

We applied both voting methods to all 34 seasons (299 weeks total) to compare their results:

\begin{enumerate}
    \item \textbf{Rank-Based Method}: Combined judge rank + fan rank, highest combined rank is eliminated
    \item \textbf{Percent-Based Method}: Combined judge percent + fan percent, lowest combined percent is eliminated
\end{enumerate}

\subsection{Overall Comparison Results}

\subsubsection{Agreement Rate}

\begin{itemize}
    \item \textbf{Total Weeks}: 299
    \item \textbf{Methods Agree}: 187 weeks (62.54\%)
    \item \textbf{Methods Disagree}: 112 weeks (37.46\%)
\end{itemize}

The high disagreement rate (37.46\%) indicates that the choice of voting method significantly impacts the results.

\subsubsection{Accuracy Comparison}

\begin{table}[H]
\centering
\caption{Accuracy Comparison of Voting Methods}
\begin{tabular}{lccc}
\toprule
\textbf{Method} & \textbf{Correct} & \textbf{Total} & \textbf{Accuracy} \\
\midrule
Rank-Based & 181 & 299 & 60.5\% \\
Percent-Based & 290 & 299 & \textbf{97.0\%} \\
\bottomrule
\end{tabular}
\end{table}

\textbf{Key Finding}: The percent-based method is significantly more accurate (97.0\% vs. 60.5\%).

\subsection{Which Method Favors Fan Votes More?}

To determine which method favors fan votes more, we analyzed cases where judge scores and fan votes disagreed:

\begin{itemize}
    \item When a contestant has low judge scores but high fan votes, the \textbf{percent-based method} is more likely to protect them from elimination
    \item The \textbf{rank-based method} gives equal weight to each rank position, while the \textbf{percent-based method} gives more weight to actual vote percentages
    \item \textbf{Conclusion}: The percent-based method favors fan votes more, as it allows high fan vote percentages to compensate for low judge scores
\end{itemize}

\subsection{Controversial Cases Analysis}

\subsubsection{Season 2: Jerry Rice}

\begin{itemize}
    \item \textbf{Background}: Runner-up despite lowest judge scores in 5 weeks
    \item \textbf{Rank Method}: Would have eliminated Jerry Rice earlier
    \item \textbf{Percent Method}: Would have protected Jerry Rice longer (matches actual result)
    \item \textbf{Conclusion}: Percent method better reflects fan support
\end{itemize}

\subsubsection{Season 4: Billy Ray Cyrus}

\begin{itemize}
    \item \textbf{Background}: 5th place despite last place judge scores in 6 weeks
    \item \textbf{Analysis}: Percent method would have protected him, rank method would have eliminated him earlier
\end{itemize}

\subsubsection{Season 11: Bristol Palin}

\begin{itemize}
    \item \textbf{Background}: 3rd place with lowest judge scores 12 times
    \item \textbf{Analysis}: Percent method consistently protected her due to high fan support
\end{itemize}

\subsubsection{Season 27: Bobby Bones}

\begin{itemize}
    \item \textbf{Background}: Won despite consistently low judge scores
    \item \textbf{Analysis}: Percent method would have protected him throughout, rank method would have eliminated him earlier
    \item \textbf{Conclusion}: Percent method allows fan popularity to overcome low judge scores
\end{itemize}

\subsection{Season 28+ Mechanism: Judges Choose from Bottom Two}

Starting in Season 28, a new mechanism was introduced: judges select which of the bottom two couples to eliminate.

\subsubsection{Impact Analysis}

\begin{itemize}
    \item This mechanism provides an additional layer of fairness
    \item It allows judges to override purely mathematical results when appropriate
    \item It reduces the impact of extreme fan vote scenarios
    \item \textbf{Recommendation}: This mechanism should be retained as it balances fan democracy with expert judgment
\end{itemize}

\subsection{Recommendations}

Based on our analysis:

\begin{enumerate}
    \item \textbf{For Future Seasons}: Use the percent-based method, as it is significantly more accurate (97.0\% vs. 60.5\%)
    \item \textbf{Retain Season 28+ Mechanism}: Keep the "judges choose from bottom two" mechanism for additional fairness
    \item \textbf{Consider Hybrid Approach}: Use percent-based method to identify bottom two, then let judges decide
\end{enumerate}

\subsection{Visualization}

Figure~\ref{fig:stage3} shows the comprehensive comparison of the two voting methods, including accuracy comparison, disagreement rates by season, and controversial case analysis.

\begin{figure}[H]
\centering
\includegraphics[width=0.9\textwidth]{visualizations/stage3_voting_comparison.png}
\caption{Stage 3: Voting Method Comparison Visualization}
\label{fig:stage3}
\end{figure}
