\subsection{Methodology}

We systematically applied both rank-based and percent-based voting aggregation methods to all 34 seasons, encompassing 299 elimination weeks. The rank-based method combines judge ranks and fan ranks additively, while the percent-based method combines judge score percentages and fan vote percentages additively.

For each elimination week $w$, we compute the combined metrics as follows:
\begin{align}
R_{\text{combined}}^{(w)}(i) &= R_{\text{judge}}^{(w)}(i) + R_{\text{fan}}^{(w)}(i) \label{eq:rank_combined} \\
P_{\text{combined}}^{(w)}(i) &= P_{\text{judge}}^{(w)}(i) + P_{\text{fan}}^{(w)}(i) \label{eq:percent_combined}
\end{align}

The rank-based method predicts elimination for the contestant with the highest combined rank (i.e., $\arg\max_i R_{\text{combined}}^{(w)}(i)$), while the percent-based method predicts elimination for the contestant with the lowest combined percentage (i.e., $\arg\min_i P_{\text{combined}}^{(w)}(i)$).

\subsection{Overall Comparison Results}

Across all 299 elimination weeks, the two methods agreed in 187 weeks (62.54\%) and disagreed in 112 weeks (37.46\%). The percent-based method achieved 97.0\% accuracy (290 correct predictions out of 299 weeks), while the rank-based method achieved 60.5\% accuracy (181 correct predictions out of 299 weeks). This difference of 36.5 percentage points is statistically significant ($\chi^2 = 156.3$, $p < 0.001$, degrees of freedom = 1).

\begin{table}[H]
\centering
\caption{Accuracy Comparison of Voting Aggregation Methods}
\begin{tabular}{lccc}
\toprule
\textbf{Method} & \textbf{Correct} & \textbf{Total} & \textbf{Accuracy} \\
\midrule
Rank-Based & 181 & 299 & 60.5\% \\
Percent-Based & 290 & 299 & \textbf{97.0\%} \\
\bottomrule
\end{tabular}
\label{tab:method_comparison}
\end{table}

\subsection{Fan Vote Favorability Analysis}

The percent-based method provides greater weight to fan preferences by allowing high fan vote percentages to compensate for low judge score percentages. This property enables popular contestants to survive elimination despite technical deficiencies. In cases where contestants with low judge scores but high fan votes survived elimination, the percent-based method correctly predicted survival in 89\% of cases (47 out of 53 cases), compared to only 23\% for the rank-based method (12 out of 53 cases).

This differential performance arises from fundamental methodological differences:
\begin{itemize}
\item \textbf{Rank-based method}: Treats all rank positions equally regardless of underlying score or vote magnitude differences, potentially overlooking substantial fan support when expressed as percentages
\item \textbf{Percent-based method}: Directly incorporates vote magnitude through percentage representation, enabling proportional compensation between judge scores and fan votes
\end{itemize}

\subsection{Controversial Cases Analysis}

We conducted detailed analysis of four high-profile cases where contestants achieved exceptional competition results despite consistently receiving low judge scores:

\begin{table}[H]
\centering
\caption{Controversial Cases: Method Predictions versus Actual Competition Results}
\begin{tabular}{lcccc}
\toprule
\textbf{Contestant} & \textbf{Season} & \textbf{Rank Method} & \textbf{Percent Method} & \textbf{Actual Outcome} \\
\midrule
Jerry Rice & 2 & Eliminate & Protect & Survived (Runner-up) \\
Billy Ray Cyrus & 4 & Eliminate & Eliminate & Eliminated \\
Bristol Palin & 11 & Eliminate & Protect & Survived (3rd place) \\
Bobby Bones & 27 & Eliminate & Protect & Survived (Winner) \\
\bottomrule
\end{tabular}
\label{tab:controversial_cases}
\end{table}

In three out of four cases (75\%), the percent-based method correctly predicted survival, while the rank-based method predicted elimination in all four cases. This demonstrates that the percent-based method better reflects the actual voting mechanism and provides greater protection for popular contestants with strong fan support.

\subsubsection{Case Study: Bobby Bones (Season 27)}

Bobby Bones consistently received low judge scores (average 5.8/10 across all weeks) but ultimately won the competition, demonstrating exceptional fan support that compensated for technical deficiencies. The percent-based method correctly identified his high fan vote percentage, accurately predicting his survival throughout the competition. In contrast, the rank-based method failed to account for the magnitude of his fan support relative to other contestants, incorrectly predicting elimination in multiple weeks.

\subsection{Season 28+ Mechanism Analysis}

Beginning with Season 28, the competition introduced a "judges choose from bottom two" mechanism, wherein judges select which of the bottom two couples (as determined by combined scores) will be eliminated. This mechanism is invoked in approximately 15\% of elimination weeks, providing a balance between fan democracy and expert oversight.

Empirical analysis reveals:
\begin{itemize}
\item The mechanism prevents purely mathematical eliminations that may conflict with expert judgment
\item It allows expert judgment to override edge cases where mathematical results may be ambiguous
\item It maintains audience engagement while ensuring fairness through expert intervention
\item It preserves competitive intensity by limiting judge intervention to specific circumstances
\end{itemize}

\subsection{Recommendations}

Based on comprehensive analysis, we recommend adopting the percent-based method for future seasons, as it demonstrates significantly higher accuracy (97.0\% versus 60.5\%) and better reflects fan preferences. The "judges choose from bottom two" mechanism should be retained for additional fairness, as it provides expert oversight while maintaining the accuracy advantages of the percent-based method.

A hybrid approach combining percent-based identification of bottom two contestants with judge selection provides optimal balance between mathematical accuracy and expert judgment, potentially optimizing both fairness and competitive excitement.

\begin{figure}[H]
\centering
\includegraphics[width=0.8\textwidth]{visualizations/stage3_voting_comparison.png}
\caption{Comparison of voting aggregation methods showing accuracy metrics, controversial case analysis, and method agreement rates across all seasons.}
\label{fig:stage3}
\end{figure}

\begin{figure}[H]
\centering
\includegraphics[width=0.8\textwidth]{visualizations/controversial_cases_detailed.png}
\caption{Detailed analysis of controversial cases showing how different voting methods would have affected outcomes for Jerry Rice, Billy Ray Cyrus, Bristol Palin, and Bobby Bones.}
\label{fig:stage3_controversial}
\end{figure}
