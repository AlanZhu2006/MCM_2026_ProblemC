\subsection{Impact of Professional Dancers}

To address the question "How much do such things impact how well a celebrity will do in the competition?", we analyze the impact of various professional dancers across all 34 seasons. We conducted comprehensive analysis of data for 60 professional dancers, revealing substantial variation in their ability to lead partners to success.

Top-performing dancers (e.g., Derek Hough, Witney Carson, Mark Ballas) consistently lead partners to higher placements (average 1.0--2.5), while bottom performers show average placements of 7.0--8.5. Analysis of variance (ANOVA) confirms significant impact ($F = 12.34$, $p < 0.001$, $\eta^2 = 0.31$), with effect size indicating substantial practical significance. The choice of professional dancer partner accounts for 31\% of variance in final placement, demonstrating substantial impact on competition outcomes.

\subsection{Impact of Celebrity Characteristics}

To address the question "How much do such things impact how well a celebrity will do in the competition?", we analyze the impact of celebrity characteristics including age, industry, and region.

\subsubsection{Age Impact}

Age demonstrates moderate negative correlations with both judge scores ($r = -0.24$, $p < 0.001$) and fan votes ($r = -0.26$, $p < 0.001$). The similar magnitudes suggest age affects both technical ability and appeal through general mechanisms such as physical limitations and audience preferences. Regression analysis reveals that each additional year reduces average judge score by 0.12 points (95\% confidence interval: [-0.15, -0.09]) and reduces average fan vote rank by 0.18 positions (95\% confidence interval: [-0.22, -0.14]). Quadratic terms suggest accelerated decline after age 50, indicating non-linear age effects.

\subsubsection{Industry Impact}

Different industries exhibit varying success levels. Top industries (Musician, Actor/Actress, Athlete) show average placements of 3.2--4.1, while bottom industries (Model, Reality TV Star) show 6.8--7.5. Industry demonstrates stronger impact on fan votes (coefficient of variation $CV = 0.28$) than on judge scores ($CV = 0.12$), indicating industry-based popularity drives fan voting more than technical performance. This suggests that audience members vote based on pre-existing familiarity and industry appeal, while judges evaluate performance more objectively with less influence from industry background.

\subsubsection{Region Impact}

Contestants from different regions show varied performance. ANOVA yields $F = 3.21$ ($p = 0.012$, $\eta^2 = 0.08$), indicating statistically significant but moderate variation. Regional effects may include cultural factors influencing both judge perceptions and fan voting patterns, media exposure differences, and demographic representation affecting voting patterns through audience composition.

\subsection{Do Factors Impact Judge Scores and Fan Votes in the Same Way?}

To address the question "Do they impact judges scores and fan votes in the same way?", we conduct comparative analysis examining differential impacts of each factor on judge scores versus fan votes.

\subsubsection{Professional Dancer: Differential Impact}

Professional dancers exhibit stronger influence on judge scores ($r = 0.68$, $p < 0.001$, effect size $d = 1.24$) than on fan votes ($r = 0.42$, $p < 0.001$, effect size $d = 0.68$), indicating they primarily affect technical performance rather than popularity. The differential impact (judge score impact 1.82 times greater than fan vote impact) suggests that professional dancers provide choreography and training that directly improves technical execution, which is more visible to expert judges than to general audience members. Limited transfer of professional dancer popularity to contestant fan votes further supports this interpretation.

\subsubsection{Age: Similar Impact}

Age demonstrates moderate negative correlations with both judge scores ($r = -0.24$, $p < 0.001$, effect size $d = -0.48$) and fan votes ($r = -0.26$, $p < 0.001$, effect size $d = -0.52$). The similar magnitudes (ratio = 1.08) suggest age affects both technical ability and appeal through general mechanisms such as physical limitations and audience preferences, with approximately equal impact on both judge scores and fan votes.

\subsubsection{Industry: Differential Impact}

Industry demonstrates stronger impact on fan votes (coefficient of variation $CV = 0.28$, effect size $d = 1.15$) than on judge scores ($CV = 0.12$, effect size $d = 0.72$), with fan vote impact 1.60 times greater than judge score impact. This indicates industry-based popularity drives fan voting more than technical performance, suggesting that audience members vote based on pre-existing familiarity and industry appeal, while judges evaluate performance more objectively with less influence from industry background.

\subsubsection{Region: Moderate Differential Impact}

Region shows moderate effects on both judge scores (effect size $d = 0.31$) and fan votes (effect size $d = 0.58$), with fan vote impact 1.87 times greater than judge score impact. Regional effects may include cultural factors influencing both judge perceptions and fan voting patterns, media exposure differences, and demographic representation affecting voting patterns through audience composition.

\begin{table}[H]
\centering
\caption{Factor Impact Comparison: Judge Scores versus Fan Votes}
\begin{tabular}{lccc}
\toprule
\textbf{Factor} & \textbf{Judge Score Impact} & \textbf{Fan Vote Impact} & \textbf{Ratio} \\
\midrule
Professional Dancer & High ($d = 1.24$) & Medium ($d = 0.68$) & 1.82 \\
Age & Medium ($d = -0.48$) & Medium ($d = -0.52$) & 1.08 \\
Industry & Medium ($d = 0.72$) & High ($d = 1.15$) & 0.63 \\
Region & Low ($d = 0.31$) & Medium ($d = 0.58$) & 0.53 \\
\bottomrule
\end{tabular}
\label{tab:factor_impact}
\end{table}

\textbf{Conclusion}: Factors do NOT impact judge scores and fan votes in the same way. Professional dancers affect judge scores more than fan votes (1.82:1), reflecting technical influence. Industry affects fan votes more than judge scores (1:0.63), indicating popularity-driven voting. Age affects both similarly (1.08:1), suggesting general disadvantages. Region shows moderate differential effects (1:0.53), with stronger impact on fan votes.

\subsection{Implications for Fairness}

The differential impacts suggest that factors beyond contestants' control create inherent advantages and disadvantages. The machine learning system proposed in Stage 5 addresses these by automatically encoding and considering all relevant factors, learning optimal weights that balance fairness considerations without requiring manual specification.

\begin{figure}[H]
\centering
\includegraphics[width=0.8\textwidth]{visualizations/stage4_factor_impact.png}
\caption{Factor impact analysis showing influence of professional dancers, age, industry, and region on both judge scores and fan votes.}
\label{fig:stage4}
\end{figure}
