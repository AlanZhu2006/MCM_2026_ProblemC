\subsection{Summary of Key Findings}

\subsubsection{Fan Vote Estimation (Stage 2)}

\begin{itemize}
    \item Successfully developed a constrained optimization model to estimate fan votes
    \item Achieved 90.0\% accuracy (265/299 weeks correctly predicted)
    \item Quantified uncertainty using Monte Carlo simulation
    \item Demonstrated that estimated fan votes are consistent with actual elimination results
\end{itemize}

\subsubsection{Voting Method Comparison (Stage 3)}

\begin{itemize}
    \item Percent-based method is significantly more accurate than rank-based method (97.0\% vs. 60.5\%)
    \item Percent-based method favors fan votes more, allowing high fan support to compensate for low judge scores
    \item 37.46\% of weeks show different results between the two methods
    \item Controversial cases (Jerry Rice, Bobby Bones) would have been handled differently by each method
    \item Season 28+ mechanism (judges choose from bottom two) provides additional fairness
\end{itemize}

\subsubsection{Factor Impact Analysis (Stage 4)}

\begin{itemize}
    \item Professional dancers significantly impact contestant performance
    \item Age negatively correlates with both judge scores (-0.24) and fan votes (-0.26)
    \item Industry and region show varying impacts on judge scores vs. fan votes
    \item Different factors affect judge scores and fan votes differently, indicating distinct evaluation criteria
\end{itemize}

\subsubsection{New Voting System (Stage 5)}

\begin{itemize}
    \item Proposed ML-based intelligent voting system achieves 97.99\% accuracy
    \item 9.03\% improvement over original system (88.96\%)
    \item Automatically considers multiple factors (age, pro dancer, industry, region)
    \item Provides better fairness and accuracy while maintaining excitement
\end{itemize}

\subsection{Recommendations for DWTS Producers}

\subsubsection{Immediate Recommendations}

\begin{enumerate}
    \item \textbf{Use Percent-Based Method}: For future seasons, use the percent-based method instead of rank-based method, as it is significantly more accurate (97.0\% vs. 60.5\%)
    
    \item \textbf{Retain Season 28+ Mechanism}: Keep the "judges choose from bottom two" mechanism, as it provides additional fairness and balances fan democracy with expert judgment
\end{enumerate}

\subsubsection{Long-term Recommendations}

\begin{enumerate}
    \item \textbf{Adopt ML-Based System}: Implement the proposed machine learning-based voting system for optimal accuracy and fairness
    
    \item \textbf{Pilot Program}: Start with a 1-2 season pilot to test the system in real-time
    
    \item \textbf{Continuous Monitoring}: Monitor system performance and update model as needed
\end{enumerate}

\subsection{Future Research Directions}

\begin{itemize}
    \item Real-time model updates using online learning
    \item Enhanced explainability using SHAP values
    \item Integration of additional data sources (social media, viewership ratings)
    \item Sensitivity analysis for model parameters
    \item Cost-benefit analysis of system implementation
\end{itemize}

\subsection{Final Thoughts}

Our analysis demonstrates that:
\begin{itemize}
    \item Fan votes can be accurately estimated from elimination results
    \item The choice of voting method significantly impacts outcomes
    \item Multiple factors influence contestant performance
    \item A data-driven approach can improve both accuracy and fairness
\end{itemize}

We believe the proposed ML-based voting system represents a significant advancement that balances accuracy, fairness, and excitement, making it an ideal solution for future DWTS seasons.
