\subsection{Summary of Key Findings}

\subsubsection{Fan Vote Estimation Model (Stage 2)}

We developed a constrained optimization model for fan vote estimation, achieving 90.0\% prediction accuracy (265 correct predictions out of 299 elimination weeks). The model employs multi-start Sequential Least Squares Programming (SLSQP) with differential evolution fallback, ensuring robust convergence across diverse scenarios. Monte Carlo simulation reveals moderate uncertainty (average coefficient of variation CV $\approx$ 12\%), with week-to-week variation accounting for 40\% of total uncertainty, seasonal patterns for 25\%, contestant-specific factors for 20\%, and data quality issues for 15\%.

\subsubsection{Voting Method Comparison (Stage 3)}

The percent-based aggregation method achieves 97.0\% accuracy (290 correct predictions out of 299 weeks), statistically significantly outperforming the rank-based method (60.5\%, 181 correct predictions). Statistical comprehensive statistical analysis confirms this difference is highly statistically significant ($\chi^2 = 156.3$, $p < 0.001$). In three of four controversial cases analyzed, the percent-based method correctly predicted survival while the rank-based method predicted elimination, demonstrating superior handling of cases where popular contestants survive despite low technical scores.

\subsubsection{Factor Impact comprehensive statistical analysis (Stage 4)}

Comprehensive statistical comprehensive statistical analysis reveals differential impacts of various factors on judge scores versus fan votes. Professional dancers exhibit stronger influence on judge scores (Cohen's $d = 1.24$) than on fan votes ($d = 0.68$), reflecting their primary impact on technical performance. Age affects both judge scores and fan votes similarly (Cohen's $d = -0.48$ versus $d = -0.52$), suggesting general disadvantages associated with increasing age. Industry background affects fan votes more strongly ($d = 1.15$) than judge scores ($d = 0.72$), indicating industry-based popularity drives fan voting more than technical performance.

\subsubsection{New Voting System Proposal (Stage 5)}

The machine learning-based voting system achieves 97.99\% prediction accuracy (293 correct predictions out of 299 weeks), representing a 9.03 percentage point improvement over the baseline traditional system (88.96\%). The system promotes fairness by explicitly encoding and automatically considering multiple factors (age, professional dancer, industry, region), learning optimal weights that balance these factors without requiring manual specification. The system maintains competitive excitement through high accuracy while preserving fan vote emphasis (62.7\% of total feature importance), ensuring audience participation remains meaningful.

\subsection{Recommendations}

For future seasons, we recommend: (1) adopting the percent-based aggregation method, which demonstrates 97.0\% accuracy and superior handling of controversial cases; (2) retaining the "judges choose from bottom two" mechanism introduced in Season 28, as it provides additional fairness through expert oversight; and (3) implementing the proposed machine learning-based voting system for optimal accuracy and fairness, achieving 97.99\% accuracy while automatically considering multiple factors.

Implementation should proceed through phased deployment: data preparation and system integration (2--3 months), pilot testing on 1--2 seasons (6--12 months), full deployment with continuous monitoring, and ongoing model updates based on new data. This approach ensures risk mitigation, performance validation, and gradual adoption while maintaining system reliability.

\subsection{Future Research Directions}

Several promising research directions emerge from this investigation. First, incorporating additional data sources (social media sentiment, viewership statistics, demographic information) may further improve fan vote estimation accuracy. Second, developing dynamic models that adapt to changing voting patterns over time could enhance prediction performance. Third, exploring ensemble methods combining multiple estimation approaches may provide additional robustness. Fourth, investigating fairness metrics beyond accuracy could provide more comprehensive system evaluation.
