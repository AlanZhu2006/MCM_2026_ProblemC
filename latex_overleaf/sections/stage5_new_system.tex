\subsection{System Design}

\subsubsection{Core Concept}

Instead of simple rank or percent addition, we propose a \textbf{Machine Learning-Based Intelligent Voting System} that:
\begin{itemize}
    \item Automatically learns the optimal way to combine judge scores and fan votes
    \item Considers multiple factors (age, pro dancer, industry, region)
    \item Adapts to different season characteristics
    \item Provides interpretable results through feature importance analysis
\end{itemize}

\subsubsection{Feature Engineering}

The system uses 12 features:

\begin{enumerate}
    \item \textbf{Basic Features} (2): Normalized judge scores and fan votes
    \item \textbf{Rank Features} (2): Normalized judge and fan ranks
    \item \textbf{Percent Features} (2): Judge and fan percentages
    \item \textbf{Relative Features} (2): Relative to group mean
    \item \textbf{Factor Features} (4): Age, professional dancer, industry, region (encoded)
\end{enumerate}

\subsubsection{Model Architecture}

\begin{itemize}
    \item \textbf{Model Type}: LightGBM (Gradient Boosting)
    \item \textbf{Input}: 12-dimensional feature vector $\mathbf{x}_i$ for each contestant $i$
    \item \textbf{Output}: Probability $P(\text{eliminated}_i | \mathbf{x}_i)$
    \item \textbf{Decision}: Contestant with highest elimination probability is predicted to be eliminated
\end{itemize}

Mathematical formulation:
\begin{equation}
\text{Eliminated} = \arg\max_{i} P(\text{eliminated}_i | \mathbf{x}_i)
\end{equation}

\subsection{Theoretical Analysis: Why More "Fair"}

\subsubsection{Fairness Definition}

We define "fairness" as: \textbf{Reducing unfairness caused by uncontrollable factors (age, professional dancer, industry, region), ensuring contestants with similar dance ability receive similar scores and vote opportunities.}

\subsubsection{Fairness Improvements}

\begin{enumerate}
    \item \textbf{Automatic Factor Consideration}: The system automatically encodes and considers age, professional dancer, industry, and region as features
    \item \textbf{Multi-dimensional Balance}: Uses 12 features instead of 2, reducing single-dimension bias
    \item \textbf{Data-Driven Weights}: Automatically learns optimal weights (fan votes: 62.7\%, judge scores: 30.4\%) instead of fixed 50\%-50\%
    \item \textbf{Adaptive Learning}: Adapts to different season characteristics automatically
\end{enumerate}

\subsubsection{Fairness Evidence}

\begin{itemize}
    \item Feature importance shows age (1.41\%) and industry (5.44\%) are considered
    \item The system balances multiple dimensions (basic, rank, percent, relative features)
    \item No single factor dominates the decision (feature importance is distributed)
\end{itemize}

\subsection{Theoretical Analysis: Why More "Exciting"}

\subsubsection{Excitement Definition}

We define "excitement" as: \textbf{Entertainment value and audience satisfaction, including prediction accuracy, controversy reduction, and competitive intensity.}

\subsubsection{Excitement Improvements}

\begin{enumerate}
    \item \textbf{Higher Accuracy}: 97.99\% vs. 88.96\% means more accurate predictions, reducing "unfair" eliminations
    \item \textbf{Reduced Controversy}: By considering multiple factors, the system reduces single-factor controversies
    \item \textbf{Maintained Competition}: Still considers fan votes (62.7\% importance), maintaining audience engagement
    \item \textbf{Professional Balance}: Considers judge scores (30.4\% importance), maintaining technical standards
\end{enumerate}

\subsection{Historical Data Validation}

\subsubsection{Model Performance}

\begin{table}[H]
\centering
\caption{ML System Performance}
\begin{tabular}{lcc}
\toprule
\textbf{Metric} & \textbf{Value} & \textbf{Interpretation} \\
\midrule
Training Accuracy & 99.65\% & High learning capability \\
Cross-Validation Accuracy & 99.56\% & Good generalization \\
Prediction Accuracy & 97.99\% & Excellent real-world performance \\
Overfitting Risk & Low & Training-validation gap: 0.09\% \\
\bottomrule
\end{tabular}
\end{table}

\subsubsection{Comparison with Original Systems}

\begin{table}[H]
\centering
\caption{System Comparison}
\begin{tabular}{lccc}
\toprule
\textbf{System} & \textbf{Accuracy} & \textbf{Correct} & \textbf{Improvement} \\
\midrule
Original (Rank/Percent) & 88.96\% & 266/299 & Baseline \\
New ML System & 97.99\% & 293/299 & +9.03\% \\
\bottomrule
\end{tabular}
\end{table}

\subsubsection{Feature Importance Analysis}

\begin{table}[H]
\centering
\caption{Top 5 Most Important Features}
\begin{tabular}{lcc}
\toprule
\textbf{Feature} & \textbf{Importance} & \textbf{Category} \\
\midrule
Fan Votes Normalized & 23.99\% & Basic \\
Fan Relative & 15.52\% & Relative \\
Fan Rank Normalized & 14.72\% & Rank \\
Judge Percent & 10.89\% & Percent \\
Fan Percent & 8.47\% & Percent \\
\bottomrule
\end{tabular}
\end{table}

\textbf{Key Insight}: Fan votes (62.7\% total importance) are more important than judge scores (30.4\% total importance), which aligns with the show's emphasis on audience engagement.

\subsection{Comprehensive Comparison}

\subsubsection{Accuracy Comparison}

The new system achieves 97.99\% accuracy, a 9.03\% improvement over the original system.

\subsubsection{Fairness Comparison}

\begin{table}[H]
\centering
\caption{Fairness Comparison}
\begin{tabular}{lcc}
\toprule
\textbf{Aspect} & \textbf{Original System} & \textbf{New System} \\
\midrule
Considers Age & No & Yes (1.41\% importance) \\
Considers Pro Dancer & No & Yes (encoded) \\
Considers Industry & No & Yes (5.44\% importance) \\
Considers Region & No & Yes (encoded) \\
Fixed Weights & Yes (50\%-50\%) & No (learned: 62.7\%-30.4\%) \\
\bottomrule
\end{tabular}
\end{table}

\subsubsection{Excitement Comparison}

\begin{table}[H]
\centering
\caption{Excitement Comparison}
\begin{tabular}{lcc}
\toprule
\textbf{Aspect} & \textbf{Original System} & \textbf{New System} \\
\midrule
Prediction Accuracy & 88.96\% & 97.99\% \\
Controversy Level & Medium & Low \\
Competition Intensity & High & High (maintained) \\
Explainability & High & Medium (with feature importance) \\
Adaptability & Low & High \\
\bottomrule
\end{tabular}
\end{table}

\subsection{Implementation Recommendations}

\subsubsection{Deployment Steps}

\begin{enumerate}
    \item \textbf{Data Preparation}: Collect historical data (already available for 34 seasons)
    \item \textbf{Model Training}: Train LightGBM model using time-series cross-validation
    \item \textbf{System Integration}: Integrate into existing voting infrastructure
    \item \textbf{Testing}: Test on new seasons with monitoring
    \item \textbf{Optimization}: Continuously update model based on new data
\end{enumerate}

\subsubsection{Resource Requirements}

\begin{itemize}
    \item \textbf{Technical}: ML model training and deployment environment
    \item \textbf{Human}: Data scientists for model maintenance
    \item \textbf{Data}: Historical and real-time contestant data
\end{itemize}

\subsubsection{Expected Outcomes}

\begin{itemize}
    \item \textbf{Short-term} (1-2 seasons): 95\%+ accuracy, reduced controversy
    \item \textbf{Long-term} (3-5 seasons): 97\%+ stable accuracy, industry benchmark
\end{itemize}

\subsubsection{Risk Assessment and Mitigation}

\begin{table}[H]
\centering
\caption{Risk Assessment}
\begin{tabular}{lcc}
\toprule
\textbf{Risk} & \textbf{Probability} & \textbf{Mitigation} \\
\midrule
Overfitting & Low & Time-series cross-validation, regularization \\
Data Dependency & Medium & Transfer learning, online learning \\
Explainability & Medium & Feature importance analysis, SHAP values \\
System Failure & Low & Backup system, monitoring \\
\bottomrule
\end{tabular}
\end{table}

\subsection{Visualization}

Figure~\ref{fig:stage5} shows the ML system's internal mechanisms, including feature importance, category importance distribution, and performance comparison.

\begin{figure}[H]
\centering
\includegraphics[width=0.9\textwidth]{visualizations/stage5_ml_system.png}
\caption{Stage 5: ML System Analysis Visualization}
\label{fig:stage5}
\end{figure}

Figure~\ref{fig:overall} provides an overall summary of all stages, including completion status, key metrics, data coverage, and system performance improvement.

\begin{figure}[H]
\centering
\includegraphics[width=0.9\textwidth]{visualizations/overall_summary.png}
\caption{Overall Summary of All Stages}
\label{fig:overall}
\end{figure}
