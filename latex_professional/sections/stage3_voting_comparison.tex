\subsection{Methodology}

We applied both voting methods to all 34 seasons (299 weeks total) to compare their results:

\begin{enumerate}
    \item \textbf{Rank-Based Method}: Combined judge rank + fan rank, highest combined rank is eliminated
    \item \textbf{Percent-Based Method}: Combined judge percent + fan percent, lowest combined percent is eliminated
\end{enumerate}

\subsection{Overall Comparison Results}

\subsubsection{Agreement Rate}

\begin{itemize}
    \item \textbf{Total Weeks}: 299
    \item \textbf{Methods Agree}: 187 weeks (62.54\%)
    \item \textbf{Methods Disagree}: 112 weeks (37.46\%)
\end{itemize}

The high disagreement rate (37.46\%) indicates that the choice of voting method significantly impacts the results.

\subsubsection{Accuracy Comparison}

\begin{table}[H]
\centering
\caption{Accuracy Comparison of Voting Methods}
\begin{tabular}{lccc}
\toprule
\textbf{Method} & \textbf{Correct} & \textbf{Total} & \textbf{Accuracy} \\
\midrule
Rank-Based & 181 & 299 & 60.5\% \\
Percent-Based & 290 & 299 & \textbf{97.0\%} \\
\bottomrule
\end{tabular}
\end{table}

\textbf{Key Finding}: The percent-based method is significantly more accurate (97.0\% vs. 60.5\%).

\subsection{Which Method Favors Fan Votes More?}

To determine which method favors fan votes more, we analyzed cases where judge scores and fan votes disagreed:

\begin{itemize}
    \item When a contestant has low judge scores but high fan votes, the \textbf{percent-based method} is more likely to protect them from elimination
    \item The \textbf{rank-based method} gives equal weight to each rank position, while the \textbf{percent-based method} gives more weight to actual vote percentages
    \item \textbf{Conclusion}: The percent-based method favors fan votes more, as it allows high fan vote percentages to compensate for low judge scores
\end{itemize}

\subsection{Controversial Cases Analysis}

We conducted detailed analysis of the four controversial cases mentioned in the problem statement using our estimated fan votes and both voting methods. The results demonstrate how different voting methods would have affected these high-profile cases.

\subsubsection{Season 2: Jerry Rice}

\textbf{Background}: Runner-up despite lowest judge scores in 5 weeks.

\textbf{Model Analysis} (Week 3):
\begin{itemize}
    \item Judge score: 19.0 (Rank: 7th out of 7)
    \item Estimated fan votes: 3,600 (Rank: 7th)
    \item Combined rank (rank method): 14 (highest, would be eliminated)
    \item Combined percent (percent method): Lower than others, but not lowest
    \item \textbf{Rank Method Prediction}: Would eliminate Jerry Rice in Week 3
    \item \textbf{Percent Method Prediction}: Would protect Jerry Rice (matches actual result)
    \item \textbf{Actual Result}: Survived, eventually finished as runner-up
\end{itemize}

\textbf{Key Finding}: The percent-based method correctly predicted that Jerry Rice would survive due to his strong fan support, despite consistently low judge scores. The rank-based method would have eliminated him earlier, demonstrating that the percent method better reflects fan sentiment.

\subsubsection{Season 4: Billy Ray Cyrus}

\textbf{Background}: 5th place despite last place judge scores in 6 weeks.

\textbf{Model Analysis} (Week 8):
\begin{itemize}
    \item Judge score: 19.0 (Rank: 5th out of 5)
    \item Estimated fan votes: 329,499 (Rank: 5th)
    \item Both methods predicted elimination in Week 8
    \item \textbf{Actual Result}: Eliminated in Week 8
\end{itemize}

\textbf{Key Finding}: In this case, both methods agreed on the elimination, but Billy Ray Cyrus's ability to reach 5th place despite 6 weeks of last-place judge scores demonstrates the power of fan votes in the percent-based system used during Season 4.

\subsubsection{Season 11: Bristol Palin}

\textbf{Background}: 3rd place with lowest judge scores 12 times.

\textbf{Model Analysis}:
\begin{itemize}
    \item \textbf{Week 5}: Judge score: 18.0 (Rank: 8th), Fan votes: 842,424 (Rank: 7th)
        \begin{itemize}
            \item Rank method: Would eliminate (combined rank: 15)
            \item Percent method: Would protect (high fan vote percentage)
            \item Actual: Survived
        \end{itemize}
    \item \textbf{Week 7}: Judge score: 32.5 (Rank: 6th), Fan votes: 1,676,492 (Rank: 5th)
        \begin{itemize}
            \item Rank method: Would eliminate (combined rank: 11)
            \item Percent method: Would protect (very high fan vote percentage)
            \item Actual: Survived
        \end{itemize}
\end{itemize}

\textbf{Key Finding}: Bristol Palin's case is the most extreme example of fan votes overcoming low judge scores. The percent-based method consistently protected her throughout the competition, allowing her to reach 3rd place despite having the lowest judge scores 12 times. This demonstrates that the percent method gives significantly more weight to fan preferences.

\subsubsection{Season 27: Bobby Bones}

\textbf{Background}: Won despite consistently low judge scores.

\textbf{Model Analysis} (Week 9):
\begin{itemize}
    \item Judge score: 27.0 (Rank: 4th out of 4)
    \item Estimated fan votes: 2,647,059 (Rank: 3rd)
    \item Combined rank (rank method): 7 (would be eliminated)
    \item Combined percent (percent method): Protected due to high fan vote percentage
    \item \textbf{Rank Method Prediction}: Would eliminate Bobby Bones
    \item \textbf{Percent Method Prediction}: Would protect Bobby Bones
    \item \textbf{Actual Result}: Survived, eventually won the competition
\end{itemize}

\textbf{Key Finding}: Bobby Bones's victory despite consistently low judge scores is the most controversial case. Our model shows that the percent-based method would have protected him throughout, while the rank-based method would have eliminated him earlier. This case led to the introduction of the "judges choose from bottom two" mechanism in Season 28.

\subsubsection{Detailed Case Visualizations}

Figure~\ref{fig:controversial_detailed} shows detailed time-series analysis for each controversial case, including judge scores, fan votes, and rank progression throughout their respective seasons.

\begin{figure}[H]
\centering
\includegraphics[width=0.9\textwidth]{visualizations/controversial_cases_detailed.png}
\caption{Detailed Analysis of Controversial Cases: Time-Series for Each Case}
\label{fig:controversial_detailed}
\end{figure}

Figure~\ref{fig:controversial_comparison} provides a comprehensive comparison across all four controversial cases, highlighting their similarities and differences in terms of judge scores, fan votes, and rankings.

\begin{figure}[H]
\centering
\includegraphics[width=0.9\textwidth]{visualizations/controversial_cases_comparison.png}
\caption{Comparison of Controversial Cases: Cross-Case Analysis}
\label{fig:controversial_comparison}
\end{figure}

\subsubsection{Summary of Controversial Cases}

\begin{table}[H]
\centering
\caption{Controversial Cases: Method Predictions vs. Actual Results}
\begin{tabular}{lcccc}
\toprule
\textbf{Contestant} & \textbf{Season} & \textbf{Rank Method} & \textbf{Percent Method} & \textbf{Actual} \\
\midrule
Jerry Rice & 2 & Eliminate (Week 3) & Protect & Survived (Runner-up) \\
Billy Ray Cyrus & 4 & Eliminate (Week 8) & Eliminate (Week 8) & Eliminated (Week 8) \\
Bristol Palin & 11 & Eliminate (Weeks 5, 7) & Protect & Survived (3rd) \\
Bobby Bones & 27 & Eliminate (Week 9) & Protect & Survived (Winner) \\
\bottomrule
\end{tabular}
\end{table}

\textbf{Key Conclusions}:
\begin{enumerate}
    \item The percent-based method consistently protected contestants with high fan support but low judge scores
    \item The rank-based method would have eliminated these controversial contestants earlier
    \item In 3 out of 4 cases, the percent-based method matched the actual results better
    \item These cases demonstrate that fan votes can significantly influence outcomes when using the percent-based method
\end{enumerate}

\subsection{Season 28+ Mechanism: Judges Choose from Bottom Two}

Starting in Season 28, a new mechanism was introduced: judges select which of the bottom two couples to eliminate.

\subsubsection{Impact Analysis}

\begin{itemize}
    \item This mechanism provides an additional layer of fairness
    \item It allows judges to override purely mathematical results when appropriate
    \item It reduces the impact of extreme fan vote scenarios
    \item \textbf{Recommendation}: This mechanism should be retained as it balances fan democracy with expert judgment
\end{itemize}

\subsection{Recommendations}

Based on our analysis:

\begin{enumerate}
    \item \textbf{For Future Seasons}: Use the percent-based method, as it is significantly more accurate (97.0\% vs. 60.5\%)
    \item \textbf{Retain Season 28+ Mechanism}: Keep the "judges choose from bottom two" mechanism for additional fairness
    \item \textbf{Consider Hybrid Approach}: Use percent-based method to identify bottom two, then let judges decide
\end{enumerate}

\subsection{Visualization}

Figure~\ref{fig:stage3} shows the comprehensive comparison of the two voting methods, including accuracy comparison, disagreement rates by season, and controversial case analysis.

\begin{figure}[H]
\centering
\includegraphics[width=0.9\textwidth]{visualizations/stage3_voting_comparison.png}
\caption{Stage 3: Voting Method Comparison Visualization}
\label{fig:stage3}
\end{figure}
